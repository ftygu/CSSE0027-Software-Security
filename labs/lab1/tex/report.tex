\documentclass{ctexart}
\usepackage{listings}
\usepackage{xcolor}
\lstset{
    basicstyle=\ttfamily,
    keywordstyle=\color{blue}\ttfamily,
    stringstyle=\color{red}\ttfamily,
    commentstyle=\color{green}\ttfamily,
    morecomment=[l][\color{magenta}]{\#}
}
\title{《漏洞利用及渗透测试基础》实验报告}
\author{张明昆  2211585}
\date{}
\begin{document}
    \maketitle
    \section{实验名称}
    IDE反汇编实验
    \section{实验目标}
    根据第二章示例2-1,在XP环境下进行VC6反汇编调试,熟悉函数调用、栈帧切换、CALL和RET指令等汇编语言实现,将call语句执行过程中的EIP变化、ESP、EBP变化等状态进行记录,解释变化的主要原因。 
    \section{实验过程}
    \subsection{进入VC反汇编}
    在VMware中下载并安装好windows xp系统,将VC6中的文件转移到虚拟机中,按照实验视频的步骤创建一个C++文件,并输入以下代码。
    \begin{lstlisting}
        #include <iostream>
        int add(int x, int y){
            int z = 0;
            z = x + y;
            return z;
        }
        void main() {
            int n = 0;
            n = add(1, 3);
            printf("%d\n", n);
        }
    \end{lstlisting}

    在n = add(1,3);处设置断点。
    按F5进入测试状态,在断点处右键选择Go to disassembly进入反汇编模式,并打开寄存器窗口,内存窗口。
    \subsection{观察add函数调用前反汇编语句、寄存器的状态}
    通过如下语句,初始化n=0,此时n的地址为[ebp-4],即EBP指针抬高4字后的位置。并将 add 函数所需要的两个参数从右向左依次压入栈中。
    \begin{lstlisting}
8:        int n = 0;
00401088   mov         dword ptr [ebp-4],0
9:        n = add(1, 3);
0040108F   push        3
00401091   push        1
    \end{lstlisting}
    \subsection{观察add函数调用前后语句}
    通过 call 指令调用 add 函数,此时call指令隐含做了两件事情,将 EIP 中的下一条指令的地址入
    栈,然后跳转到 add 函数所在的代码块中。
    \begin{lstlisting}
00401093   call        @ILT+0(add) (00401005)      
    \end{lstlisting}
    在 add 函数调用后,进行栈平衡操作,将 ESP 恢复到调用 add 函数之前的状态,然后将
    函数返回值从 EAX 寄存器中转移到局部变量 n 所在的地址处。
    \begin{lstlisting}
00401098   add         esp,8 
0040109B   mov         dword ptr [ebp-4],eax
    \end{lstlisting}
    \subsection{add函数内部栈帧切换}
    在 call 指令执行时,发生了两件事情,将 EIP 中的返回地址入栈,ESP-4,设置
    EIP 的值,实现从 main 函数到 add 函数的跳转。
    \begin{lstlisting}
        EAX = CCCCCCCC EBX = 7FFD6000
        ECX = 00000000 EDX = 003C0DD8
        ESI = 00000000 EDI = 0012FF80
        EIP = 00401030 ESP = 0012FF24
        EBP = 0012FF80 EFL = 00000216 
    \end{lstlisting}
    然后将 EBP 栈底指针上提到 ESP 处,将 ESP-44h,开辟 add 函数的栈帧。
    \begin{lstlisting}
        EAX = CCCCCCCC EBX = 7FFD6000
        ECX = 00000000 EDX = 003C0DD8
        ESI = 00000000 EDI = 0012FF80
        EIP = 00401036 ESP = 0012FEDC
        EBP = 0012FF20 EFL = 00000212         
    \end{lstlisting}
    再将原始的寄存器信息入栈保存,将 EBX,ESI,EDI 入栈。
    再对局部变量进行初始化。
    \begin{lstlisting}
        EAX = CCCCCCCC EBX = 7FFD6000
        ECX = 00000000 EDX = 003C0DD8
        ESI = 00000000 EDI = 0012FF20
        EIP = 0040104F ESP = 0012FED0
        EBP = 0012FF20 EFL = 00000212        
    \end{lstlisting}
    在 add 函数执行完成之后,return z;这时将函数的返回值从局部变量中转移到寄存器
    EAX 中。
    \begin{lstlisting}
        EAX = 00000004 EBX = 7FFD6000
        ECX = 00000000 EDX = 003C0DD8
        ESI = 00000000 EDI = 0012FF20
        EIP = 0040105B ESP = 0012FED0
        EBP = 0012FF20 EFL = 00000202       
    \end{lstlisting}
    随后要清理函数开辟的栈帧,将原始的寄存器信息出栈,恢复状态。
    通过 mov esp,ebp 清除函数开辟的栈帧。
    \begin{lstlisting}
        EAX = 00000004 EBX = 7FFD6000
        ECX = 00000000 EDX = 003C0DD8
        ESI = 00000000 EDI = 0012FF80
        EIP = 00401060 ESP = 0012FF20
        EBP = 0012FF20 EFL = 00000202        
    \end{lstlisting}
    然后将原始栈帧的栈底EBP的值出栈,恢复了原始的 EBP 状态。
    \begin{lstlisting}
        EAX = 00000004 EBX = 7FFD6000
        ECX = 00000000 EDX = 003C0DD8
        ESI = 00000000 EDI = 0012FF80
        EIP = 00401061 ESP = 0012FF24
        EBP = 0012FF80 EFL = 00000202        
    \end{lstlisting}
    ret 指令完成了两件事情,将当前的栈顶 ESP 中保存原始 EIP 的值出栈到 EIP 中,然后跳回到调用函数的代码块中。
    \section{心得体会}
    通过本次实验,我更加直观清晰地了解了函数调用时栈帧、地址、EIP、ESP、EBP等指针存放状态的变化,熟悉了反汇编、断点、单步执行等操作。
    我理解了栈帧调整的具体步骤,即:返回地址入栈,参数从右向左入栈,开辟函数空间,之后再相对应地释放函数空间,读取返回地址,最终返回到主函数。
    这为日后的汇编学习与软件安全的学习打下基础。
\end{document}